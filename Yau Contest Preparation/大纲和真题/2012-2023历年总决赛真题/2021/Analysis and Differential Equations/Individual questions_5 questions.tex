

\documentclass[12pt]{article}
\usepackage{amssymb}
\usepackage{amsmath}
\usepackage{graphics}
\usepackage{epsfig}

\textwidth =15.6cm \textheight=22cm \hoffset0.5cm \voffset-1.3cm

\baselineskip=5mm
%\renewcommand{\baselinestretch}{1.3}
\renewcommand{\arraystretch}{1.5}
\oddsidemargin 0pt \evensidemargin 0pt
\parskip 0.1cm

\newtheorem{Assumption}{Assumption}[part]
\newtheorem{Corollary}{Corollary}[part]
\newtheorem{Definition}{Definition}[part]
\newtheorem{Example}{Example}[part]
\newtheorem{Lemma}{Lemma}[part]
\newtheorem{Proposition}{Proposition}[part]
\newtheorem{Remark}{Remark}[part]
\newtheorem{Theorem}{Theorem}[part]

\renewcommand{\theAssumption}{\thesection.\arabic{Assumption}}
\renewcommand{\theCorollary}{\thesection.\arabic{Corollary}}
\renewcommand{\theDefinition}{\thesection.\arabic{Definition}}
%\renewcommand{\theequation}{\thesection.\arabic{equation}}
\renewcommand{\theExample}{\thesection.\arabic{Example}}
\renewcommand{\theLemma}{\thesection.\arabic{Lemma}}
\renewcommand{\theProposition}{\thesection.\arabic{Proposition}}
\renewcommand{\theRemark}{\thesection.\arabic{Remark}}
\renewcommand{\theTheorem}{\thesection.\arabic{Theorem}}

%\numberwithin{Assumption}{section} \numberwithin{Corollary}{section}
%\numberwithin{Definition}{section} \numberwithin{equation}{section}
%\numberwithin{Example}{section} \numberwithin{Lemma}{section}
%\numberwithin{Proposition}{section} \numberwithin{Remark}{section}
%\numberwithin{Theorem}{section}

\def\E{\mathbb E}


%\pagestyle{empty}
\begin{document}

\begin{center}
12th Oral Exam of S.-T. Yau College Student Mathematics Contests 2021\\

\vspace{0.1cm}

\Large {\bf Analysis and differential equations}\\

\vspace{0.1cm}

{\bf Individual Contest}

($\ast$ The forth and fifth problems are optional)

\end{center}

\vskip 1cm

\begin{itemize}

\item[1.]  Let $\mathbb{Z}$ be the set of integers. Recall that $\vec{a}:=\{a_{j}\}_{j\in \mathbb{Z}}\in l^{q}(\mathbb{Z})$ if and only if $(\sum_{j\in \mathbb{Z}}|a_{j}|^{q})^{\frac{1}{q}}<\infty$. Answer the following questions and justify your answers:
    \begin{itemize}
    \item[(i)] Is the embedding $l^{2}(\mathbb{Z})\rightarrow l^{4}(\mathbb{Z})$ continuous ?
    \item[(ii)] Is the embedding $l^{2}(\mathbb{Z})\rightarrow l^{4}(\mathbb{Z})$ compact?
    \item[(iii)] Is the embedding $l^{2}(\mathbb{Z})\rightarrow l^{4}(\mathbb{Z})$ compact modulo translation? More precisely, let $\{\vec{a}_{n}\}_{n}$ be a sequence in $l^{2}(\mathbb{Z})$ with $\vec{a_{n}}=\{a_{j,n}\}_{j\in \mathbb{Z}}$. Can one find $k_{n}\in \mathbb{Z}$ for $n\geq 1$, such that $\{\vec{b}_n\}_n$, with $\vec{b}_{n}:=\{b_{j,n}\}_{j\in\mathbb{Z}}$ and $b_{j,n}=a_{j-k_{n}, n}$ has a convergent subsequence in $l^{4}(\mathbb{Z})$?
    \end{itemize}



\vskip 1cm

\item[2.]

Let $S^1$ be the unit circle in $\mathbb{R}^2$. For any $v\in S^1$, let $\pi_v: \mathbb{R}^2 \rightarrow \mathbb{R}: \pi_v(x) = \langle v,x\rangle$. Let $A,B$ be two bounded open convex sets in $\mathbb{R}^2$ and $\lambda_1$ be the Lebesgue measure on $\mathbb{R}^1$.
    \begin{itemize}
    \item[(i)] If for all $v\in S^1$, $\lambda_1(\pi_v(A)) = \lambda_1(\pi_v(B))$, can one conclude that $A = B$? Can one conclude that $A=B$ modulo isometries?
    \item[(ii)] If for all $v\in S^1$, $\pi_v(A) = \pi_v(B)$, can one say that $A = B$?
    \end{itemize}
Justify your answers.

\vskip 1cm

\item[3.] Assume that  $\rho\in C^1_0 (\mathbb{R}^3)$  satisfies $\rho(x)\ge 0$ for $x\in \mathbb{R}^3$ and
\[ \nabla (\rho^{4/3})+\rho \nabla \phi=0, \ on \  \mathbb{R}^3 \]
where \[\phi(x)=-\int_{\mathbb{R}^3}\frac{\rho(y)}{|x-y|}dy. \]
Evaluate the integral
\[ \int_{\mathbb{R}^3}(3\rho^{4/3}+\frac{1}{2}\rho\phi)dx,\]
and justify your answer.


\newpage

\item[$4^\ast$.] Let $D = \{z, |z| < 1\}$. Determine $Aut(D)$, the group of holomorphic automorphisms of the unit disk.


\vskip 1cm

\item[$5^\ast$.] Assume that $H$ is space with inner product, and $x_{k} \overset{w}{\to}x\ (k\to \infty)$. Prove that there exists a subsequence $\{x_{k_{n}}\} \subset \{x_k\}$ such that $\frac{x_{k_{1}}+x_{k_{2}}+\cdots+x_{k_{n}}}{n}\to x(n\to \infty)$.


\end{itemize}

\end{document}
