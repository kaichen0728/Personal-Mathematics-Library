\documentclass[a4paper,12pt,leqno]{article}
\usepackage[margin=3cm]{geometry}
\usepackage[all]{xy}
%\usepackage{mathtools}
\usepackage{colonequals}
\usepackage{amssymb}
\usepackage{amsxtra}
\usepackage{amsthm}
\usepackage{mathdots}
%\usepackage[notref,notcite]{showkeys}
\usepackage[T1]{fontenc}
\usepackage{lmodern}
\usepackage[pagebackref,breaklinks]{hyperref}
\usepackage[shortlabels]{enumitem}
\setlist[enumerate,1]{(a)}
\setlist[enumerate,2]{(1)}
%\setlist{nosep}
\theoremstyle{definition}
\newtheorem{prob}{Problem}
\newtheorem{tprob}{Problem}
\newtheorem*{remark}{Remark}

\newcommand{\C}{\mathbb{C}}
\newcommand{\FF}{\mathbb{F}}
\newcommand{\QQ}{\mathbb{Q}}
\newcommand{\RR}{\mathbb{R}}
\newcommand{\ZZ}{\mathbb{Z}}
\newcommand{\cO}{\mathcal{O}}
\newcommand{\fm}{\mathfrak{m}}
\newcommand{\fn}{\mathfrak{n}}
\newcommand{\legendre}[2]{\genfrac{(}{)}{}{}{#1}{#2}}
 

\begin{document}
\begin{center}
\begin{center}
 { \large  S.-T. Yau College Student Mathematics Contests 2022}
 
 \

  { \bf \Large Algebra and Number Theory   } 
 \medskip 
  
  {\bf \large (Overall individual round) }
\end{center}
\end{center}

\bigskip

%\subsection*{Suggested problems for Algebra and Number Theory}
%\begin{prob}Consider the Abelian group $A=(\ZZ/3\ZZ)^4$.
%\begin{enumerate}
%\item Find the number of subgroups of $A$.
%\item Find the number of composition series of $A$. (Recall that a
%    \emph{composition series} of $A$ is a sequence of subgroups
%    $0=A_0\subseteq A_1\subseteq \dots \subseteq A_n=A$ such that
%    $A_{i}/A_{i-1}$ is a simple group for every $0<i\le n$.)
%\end{enumerate}
%\end{prob}
%
%\begin{proof}[Solution]
%We will solve the problem more generally for $A=\FF_p^n$, where $p$ is a
%prime number and $\FF_p=\ZZ/p\ZZ$.
%
%(a) For $0\le m\le n$, there is a bijection between the set
%$\mathrm{Gr}(n,m)$ of subgroups of $A$ of order $p^m$ and
%$M_m/\mathrm{GL}_m(\FF_p)$, where $M_m$ is the set of $n\times m$ matrices
%with entries in $\FF_p$ of rank $m$. Thus
%\[\#\mathrm{Gr}(n,m)=\frac{\prod_{i=0}^{m-1}(p^n-p^i)}{\prod_{i=1}^{m-1}(p^m-p^i)}=\frac{\prod_{i=0}^{m-1}(p^{n-i}-1)}{\prod_{i=0}^{m-1}(p^{m-i}-1)}.\]
%In the case $p=3$, $n=4$, there are $212$ subgroups of $A$: $1$ of order
%$1$, $40$ of order $3$, $130$ of order $9$, $40$ of order $27$, and $1$ of
%order $81$.
%
%(b) Let $B\subseteq G=\mathrm{GL}_n(\FF_p)$ be the subgroup of upper
%triangular matrices. The number of composition series equals
%\[\#(G/B)=\frac{\prod_{i=0}^{n-1}(p^n-p^i)}{(p-1)^n\prod_{i=0}^{n-1}p^i}=\prod_{i=0}^{n-2}\frac{p^{n-i}-1}{p-1}.\]
%In the case $p=3$, $n=4$, there are $40\times 13\times 4=2080$ composition
%series.
%\end{proof}


 
\begin{prob}
Prove that $\mathbb{C}[x,y]/(x^2+y^2-1)$ is a unique factorization domain (UFD), but $\mathbb{R}[x,y]/(x^2+y^2-1)$ is not a UFD.
\end{prob}
\begin{proof}
Let $u=x+iy$, $v=x-iy$, then \[\mathbb{C}[x,y]/(x^2+y^2-1)\cong \mathbb{C}[u,v]/(uv-1)\cong \mathbb{C}[u,u^{-1}],\]
which is an Euclidean domain.

It is easy to see that \[\mathbb{C}[x,y]/(x^2+y^2-1)\cong \mathbb{C}[\cos t,\sin t],\]
\[\mathbb{R}[x,y]/(x^2+y^2-1)\cong\mathbb{R}[\cos t,\sin t].\]
Prove that 
\[\cos t=\frac{1}{2}(\cos t+i \sin t-i)(\sin t+i \cos t+1)\]
\[1-\sin t=\frac{1}{2}(\cos t+i \sin t-i)(\cos t-i \sin t+i)\]
are irreducible factorizations in $\mathbb{C}[\cos t,\sin t]$. Using this fact, prove that  
\[\cos^2t=\cos t\cdot \cos t=(1-\sin t)(1+\sin t)\]
are two different irreducible factorizations in $\mathbb{R}[x,y]/(x^2+y^2-1)$. 
\end{proof}
 
 \medskip

\begin{prob}
Consider the equation $f(x)=x^3-x-1\in \mathbb{Z}[x]$. Let $p\neq 23$ be a prime number.

\begin{itemize}
\item[(a)] Prove that $f(x)=0\ (\mathrm{mod}\ p)$ has exactly one root in $\mathbb{F}_p$ if and only if $\legendre{p}{23}=-1$. 

\item[(b)] Let $K/\mathbb{Q}$ be the Galois closure corresponding to $f$. For $p$ as in (a), determine the values of $e,f,g$ in the prime decomposition of $p\mathcal{O}_K$.  
\end{itemize}.
\end{prob}

\begin{proof}: (a) The discriminant of $f$ is equal to $-4\times (-1)^3-27\times (-1)^2=-23$, so for $p\neq 23$, $f\ (\mathrm{mod}\ p)$ has no multiple roots. Let $\alpha_1,\alpha_2,\alpha_3$ be the  roots of $f$ in $\overline{\mathbb{F}}_p$. 

First assume $p=2$. Then $f\ (\mathrm{mod}\ 2)=0$ has no solution. Since $\legendre{2}{23}=(-1)^{\frac{23^2-1}{8}}=1$, the claim holds in this case.

Assume $p\neq 2$. Consider the product
\[\Delta:=(\alpha_1-\alpha_2)(\alpha_1-\alpha_3)(\alpha_2-\alpha_3).\]
\begin{itemize}
\item If all of $\alpha_i$ lie in $\mathbb{F}_p$, then clearly $\Delta\in\mathbb{F}_p$. 
\item If only one of $\alpha_i$ lies in $\mathbb{F}_p$, then $\Delta$ is not fixed by  $\mathrm{Gal}(\overline{\mathbb{F}}_p/\mathbb{F}_p)$ (as $p\neq 2$), so $\Delta\notin\mathbb{F}_p$.
\item If none of $\alpha_i$ lies in $\mathbb{F}_p$, then $f(x)\ \mathrm{mod}\ p$ is irreducible, and corresponds to a degree $3$ extension of $\mathbb{F}_p$. Again it is easy to see that any $\mathrm{Gal}(\mathbb{F}_{p^3}/\mathbb{F}_p)$ fixes $\Delta$, thus $\Delta\in \mathbb{F}_p$. 
\end{itemize}
Note that $\Delta^2=-23$.  
Thus, $f(x)=0\ (\mathrm{mod}\ p)$ has exactly one root if and only if $-23$ is not a square in $\mathbb{F}_p$, i.e. $\legendre{-23}{p}=-1$. By quadratic reciprocity, this is further equivalent to $\legendre{p}{23}=-1$. 
 
(b)  It is easy to see that $[K:\mathbb{Q}]=6$ and $\mathrm{Gal}(K/\mathbb{Q})=S_3$.  
Let $\beta$ be a root of $f(x)$ and $K_1=\mathbb{Q}(\beta)\subset K$.  It is easy to see that  $\mathcal{O}_{K_1}=\mathbb{Z}[\beta]$ (as the discriminant of $f$ is square-free). 

First determine the decomposition of $p\cO_{K_1}$.  By assumption, we have   $f\ (\mathrm{mod}\ p)=(x-\alpha)(\mathrm{degree }\ 2\ \mathrm{term})$, thus   $p\cO_{K_1}=\mathfrak{p}_1\mathfrak{p}_2$ with $f(\mathfrak{p}_2|p)=2$. This implies   $g\geq 2$ and $2|f$, hence \[e=1,\ \ f=2,\ \  g=3.\]   
\end{proof}

 \end{document}
