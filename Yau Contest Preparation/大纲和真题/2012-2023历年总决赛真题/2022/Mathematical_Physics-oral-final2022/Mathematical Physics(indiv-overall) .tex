\documentstyle[12pt]{article}
%\documentclass[prd,aps,preprint]{revtex}
\renewcommand{\baselinestretch}{1.2}
%\newcommand{\sect}[1]{\setcounter{equation}{0}\section{#1}}
%\renewcommand{\theequation}{\thesection.\arabic{equation}}
\textwidth 160mm
\textheight 240mm
\newcommand{\bea}{\begin{eqnarray}}
\newcommand{\eea}{\end{eqnarray}}
\newcommand{\be}{\begin{equation}}
\newcommand{\ee}{\end{equation}}
\newcommand{\vs}[1]{\vspace{#1 mm}}
\newcommand{\hs}[1]{\hspace{#1 mm}}
\renewcommand{\a}{\alpha}
\renewcommand{\b}{\beta}
\renewcommand{\c}{\gamma}
\renewcommand{\d}{\delta}
\newcommand{\e}{\epsilon}
\newcommand{\dsl}{\pa \kern-0.5em /}
\newcommand{\la}{\lambda}
\newcommand{\half}{\frac{1}{2}}
\newcommand{\pa}{\partial}
\renewcommand{\t}{\theta}
\newcommand{\tb}{{\bar \theta}}
\newcommand{\nn}{\nonumber\\}
\newcommand{\p}[1]{(\ref{#1})}
\newcommand{\lan}{\langle}
\newcommand{\ran}{\rangle}

\begin{document}
\topmargin 0pt
\oddsidemargin 0mm

%\renewcommand{\thefootnote}{\fnsymbol{footnote}}

%\begin{titlepage}

%\begin{flushright}

%USTC-ICTS-05-8\\

%MCTP-05-?\\

%hep-th/yymmnnn\\

%SINP-TNP/02-7

%\end{flushright}
%\vspace{6mm}

\begin{center}
 \bf

Mathematical Physics \\
(Individual Overall Contest) \\
%(Time: 2 Hours)} \vs{6}

%Name:$\underline{\qquad\qquad\qquad\qquad\qquad\qquad}$ Student
%ID:$\underline{\qquad\qquad\qquad\qquad}$

\end{center}

\vspace{4mm}
\renewcommand{\a}{\alpha}
\renewcommand{\b}{\beta}
\renewcommand{\c}{\gamma}
\renewcommand{\d}{\delta}
\topmargin 0pt \oddsidemargin 0mm
{\small
\noindent {\bf Prob. 1}
Consider a scalar in a D-dimensional spacetime background, the gravity and the scalar classical dynamics is described by the following total action
\be\label{ta}
S = \frac{1}{16 \pi G_{D}} \int d^{D} x \sqrt{- g} R - \frac{1}{2} \int d^{D} x \sqrt{- g} g^{\alpha\beta} \partial_{\alpha} \phi \partial_{\beta} \phi,
\ee
where $G_{D}$ is the D-dimensional Newton constant,  $g = \det g_{\alpha\beta}$ and $g^{\alpha\beta}$ is the inverse of $g_{\alpha\beta}$, i.e., $g_{\alpha\gamma} g^{\gamma\beta} = \delta_{\alpha}^{\beta}$.\\
\noindent
\noindent
1)  Derive the Einstein equation and the equation of motion for the scalar field, respectively;\\
\noindent 
2) In any dimension, the Riemann tensor obeys
\be
R_{\alpha\beta\gamma\delta} = R_{\gamma\delta\alpha\beta} =  - R_{\beta\alpha\gamma\delta} = - R_{\alpha\beta\delta\gamma}.
\ee
Now in two dimensions, these relations imply a connection between the Riemann tensor and the Ricci scalar $R$. Find this precise relation;\\
\noindent
3) Using this relation, what is the implication of  the Einstein equation obtained in 1) for D = 2 ? 

\medskip
\noindent{\bf Prob. 2}
Consider the four-dimensional U(1) gauge theory: ${\cal L}= -{1\over4} F_{\mu\nu} F^{\mu\nu}$, where $F_{\mu\nu} = \partial_\mu A_\nu - \partial_\nu A_\mu$. One can construct a class of gauge invariant local operators in the following form
\begin{equation}
{\cal O}_{2n}(x) \sim (\prod_{n\textrm{-pairs}} \eta^{\mu_i \mu_j} ) (\partial_{\mu_1} \partial_{\mu_2} \ldots \partial_{\mu_{m-2}}F_{\mu_{m-1} \mu_{m}}) (\partial_{\mu_{m+1}} \ldots \partial_{\mu_{2n-2}}  F_{\mu_{2n-1}\mu_{2n}})(x) \,, \nonumber
%= i \begin{pmatrix} {\bar\psi_u}  & {\bar\psi_d} \end{pmatrix} {\displaystyle{\not}}{D} \begin{pmatrix} \psi_u  \\ \psi_d \end{pmatrix} - {1\over4} (F^a_{\mu\nu})^2  \,,
\end{equation}
which contains 2 field strength tensor $F_{\mu\nu}$ and $2n-4$ derivatives $\partial_\mu$. All $2n$ Lorentz indices are contracted in $n$ pairs, thus every operator is a Lorentz scalar.
By contracting Lorentz indices in different ways, you may write down different operators for a given $n$.\\

\noindent
Let $A_\mu$ be on-shell physical fields in the operator, and solve the following problems:\\
\noindent
(1) %On-shell condition of $A_\mu$ implies that $F_{\mu\nu}$ satisfies the equation of motion. 
Derive all the equations that $F_{\mu\nu}$ should satisfy.  \\
%({\color{blue} Check that $F^{\mu\nu} \simeq p^\mu \epsilon^\nu - p^\nu \epsilon^\mu$ satisfies these equations.})\\
(2) In the $n=2$ case, it should be easy to see that there is only one operator  $F_{\mu\nu} F^{\mu\nu}$.
Show that, for all possible ways of Lorentz contractions, there is only one independent operator for the $n = 3$ case. \\
(3)  Derive the set of independent operators for the $n=4$ case. \\
(4) Derive the set of independent operators for general $n$.\\
\noindent 
You may find it convenient to use a short notation for Lorentz indices such that $F_{\mu\nu}F^{\mu\nu} = F_{12} F_{12}$, and $\partial_\rho F^{\mu\nu} \partial^\mu F_{\nu\rho} = \partial_1 F_{23} \partial_2 F_{31}$.

}
 \end{document}
