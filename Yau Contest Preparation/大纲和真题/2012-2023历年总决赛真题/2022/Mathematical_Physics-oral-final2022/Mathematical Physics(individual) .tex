\documentstyle[12pt]{article}
%\documentclass[prd,aps,preprint]{revtex}
\renewcommand{\baselinestretch}{1.2}
%\newcommand{\sect}[1]{\setcounter{equation}{0}\section{#1}}
%\renewcommand{\theequation}{\thesection.\arabic{equation}}
\textwidth 160mm
\textheight 240mm
\newcommand{\bea}{\begin{eqnarray}}
\newcommand{\eea}{\end{eqnarray}}
\newcommand{\be}{\begin{equation}}
\newcommand{\ee}{\end{equation}}
\newcommand{\vs}[1]{\vspace{#1 mm}}
\newcommand{\hs}[1]{\hspace{#1 mm}}
\renewcommand{\a}{\alpha}
\renewcommand{\b}{\beta}
\renewcommand{\c}{\gamma}
\renewcommand{\d}{\delta}
\newcommand{\e}{\epsilon}
\newcommand{\dsl}{\pa \kern-0.5em /}
\newcommand{\la}{\lambda}
\newcommand{\half}{\frac{1}{2}}
\newcommand{\pa}{\partial}
\renewcommand{\t}{\theta}
\newcommand{\tb}{{\bar \theta}}
\newcommand{\nn}{\nonumber\\}
\newcommand{\p}[1]{(\ref{#1})}
\newcommand{\lan}{\langle}
\newcommand{\ran}{\rangle}

\begin{document}
\topmargin 0pt
\oddsidemargin 0mm

%\renewcommand{\thefootnote}{\fnsymbol{footnote}}

%\begin{titlepage}

%\begin{flushright}

%USTC-ICTS-05-8\\

%MCTP-05-?\\

%hep-th/yymmnnn\\

%SINP-TNP/02-7

%\end{flushright}
%\vspace{6mm}

\begin{center}
 \bf

Mathematical Physics\\
 (Individual Contest) \\
%(Time: 2 Hours)} \vs{6}

%Name:$\underline{\qquad\qquad\qquad\qquad\qquad\qquad}$ Student
%ID:$\underline{\qquad\qquad\qquad\qquad}$

\end{center}

\vspace{4mm}
\renewcommand{\a}{\alpha}
\renewcommand{\b}{\beta}
\renewcommand{\c}{\gamma}
\renewcommand{\d}{\delta}
\topmargin 0pt \oddsidemargin 0mm
{\small
\noindent {\bf Prob. 1} { The Hellmann-Feynman theorem:  Given a Hamiltonian $H$ which has discrete  energy levels and smoothly depends on a coupling
parameter $\gamma$, define ${\mathcal A}=\frac{\partial H}{\partial \gamma} $.
\begin{itemize}
\item For every eigenstate $|\Psi_j\rangle$ of $H$ with its energy eigenvalue $E_j$, prove that $\langle \Psi_j |{\mathcal A} |\Psi_j\rangle
=\frac{\partial E_j}{\partial \gamma}. $
\item Taking as an example the one-dimensional quantum oscillator described by the Hamiltonian $ H=-\frac{\partial^2}{\partial x^2}+\frac{\omega^2 x^2}{4},\quad \omega\in\mathbb{R}\quad{\rm and}\,\,\omega>0,$
show  that
$$\langle\Psi_j|x^2 |\Psi_j\rangle =\frac{(2j+1)}{\omega}.$$
\end{itemize}


\medskip
\noindent{\bf Prob. 2}
You are given the weak-field Newtonian limit of space-time as \be ds^2 = - (1 + 2 \phi) dt^2 +
dx^2 + dy^2 + dz^2,\ee and the Newtonian gravitational potential
$\phi$ satisfying $ {\vec \nabla}^2 \phi = 4 \pi G \rho $  (i.e., $ {\vec \nabla}^2 g_{00}= - 8 \pi G \rho $) with ${\vec \nabla}^2 = \partial^{2}_{x} + \partial^{2}_{y} + \partial^{2}_{z}$,  $G$ the Newton
constant and $\rho$ the matter density.  Based on this information and the principle of general covariance, derive the complete Einstein
field equation. You are given \bea R^\rho\,_{\mu \sigma \nu} &=&
\partial_\sigma \Gamma^\rho_{\mu\nu} - \partial_\nu
\Gamma^\rho_{\sigma\mu} + \Gamma \Gamma - {\rm terms}\nn
\Gamma^{\rho}_{\mu\nu} &=& \frac{1}{2} g^{\rho\sigma}\left( \partial_{\mu} g_{\sigma\nu} + \partial_{\nu} g_{\mu\sigma} - \partial_{\sigma} g_{\mu\nu}\right).\eea
\\

\medskip
\noindent{\bf Prob.3}  Consider a 2-form field in  6-dimensional spacetime
\be
B_{(2)}=\frac12 \,B_{\mu\nu} d x^{\mu}\wedge dx^{\nu}, \qquad \mu,\nu=0,1,\cdots,5.
\ee
The free theory action is
\be
S=-\int{\rm d}^6x\,\left(\frac1{12} H^{\mu\nu\rho}H_{\mu\nu\rho}+\frac14m^2B^{\mu\nu}B_{\mu\nu}\right),
\ee
where $H$ is the corresponding 3-form field strength associated with the 2-form $B_{(2)}$
\be
H_{(3)}=\frac16 \, H_{\mu\nu\rho} d x^{\mu}\wedge d x^{\nu}\wedge d x^{\rho}=\frac12 \partial_{[\mu}B_{\nu\rho]} d x^{\mu}\wedge d x^{\nu}\wedge d x^{\rho}= d B_{(2)}\,.\nn
\ee
(a)In the massive case, count the on-shell propagating degrees of freedom  and describe the corresponding representation of Poincar\'e group.
\\(b)In the massless case, count the physical on-shell propagating degrees of freedom by eliminating the gauge redundancies, and describe the corresponding representation of Poincar\'e group.}}

}
 \end{document}
