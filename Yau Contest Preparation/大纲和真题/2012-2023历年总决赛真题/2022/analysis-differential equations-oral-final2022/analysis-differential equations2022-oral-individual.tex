

\documentclass[12pt]{article}
\usepackage{amssymb}
\usepackage{amsmath}
\usepackage{graphics}
\usepackage{epsfig}

\textwidth =15.6cm \textheight=22cm \hoffset0.5cm \voffset-1.3cm

\baselineskip=5mm
%\renewcommand{\baselinestretch}{1.3}
\renewcommand{\arraystretch}{1.5}
\oddsidemargin 0pt \evensidemargin 0pt
\parskip 0.1cm

\newtheorem{Assumption}{Assumption}[part]
\newtheorem{Corollary}{Corollary}[part]
\newtheorem{Definition}{Definition}[part]
\newtheorem{Example}{Example}[part]
\newtheorem{Lemma}{Lemma}[part]
\newtheorem{Proposition}{Proposition}[part]
\newtheorem{Remark}{Remark}[part]
\newtheorem{Theorem}{Theorem}[part]

\renewcommand{\theAssumption}{\thesection.\arabic{Assumption}}
\renewcommand{\theCorollary}{\thesection.\arabic{Corollary}}
\renewcommand{\theDefinition}{\thesection.\arabic{Definition}}
\renewcommand{\theequation}{\thesection.\arabic{equation}}
\renewcommand{\theExample}{\thesection.\arabic{Example}}
\renewcommand{\theLemma}{\thesection.\arabic{Lemma}}
\renewcommand{\theProposition}{\thesection.\arabic{Proposition}}
\renewcommand{\theRemark}{\thesection.\arabic{Remark}}
\renewcommand{\theTheorem}{\thesection.\arabic{Theorem}}

%\numberwithin{Assumption}{section} \numberwithin{Corollary}{section}
%\numberwithin{Definition}{section} \numberwithin{equation}{section}
%\numberwithin{Example}{section} \numberwithin{Lemma}{section}
%\numberwithin{Proposition}{section} \numberwithin{Remark}{section}
%\numberwithin{Theorem}{section}

\def\E{\mathbb E}


\pagestyle{empty}
\begin{document}

\begin{center}
S.-T. Yau College Student Mathematics Contests 2022\\

\vspace{0.1cm}

\Large {\bf Analysis and differential equations individual}

\vspace{0.1cm}

%\large {\bf Individual (5 problems)}

%\vspace{0.1cm}


\end{center}

\vskip 1cm


\noindent {\bf Problem 1.} Assume U is a bounded smooth open set, $q\in[1,\infty)$ and
%\be
$f_{k}\rightharpoonup f \quad\mbox{weakly in } L^q(U)$
%\ee
and
%\be
$f_{k}\rightarrow f\quad\mbox{a.e in U},$
%\ee
then
%\be
$\lim_{k\rightarrow \infty}\big(\|f_k\|_{L^q(U)}-\|f_k-f\|_{L^q(U)}\big)=\|f\|_{L^q(U)}$.
%\ee

\bigskip

\noindent {\bf Problem 2.} Let $\Omega$ be a proper (nonempty and $\ne\mathbb{C}$) region of $\mathbb{C}$ which is simply connected. Let $\mathbb{D}$ be the unit disc and $z_0\in\Omega$, and
$$\mathcal{F}=\{f| f:\Omega\to\mathbb{D} \text{ holomorphic, injective and $f(z_0)=0$}\}$$
One strategy of proving the (existence part of the) Riemann mapping theorem is that the desired map $f$ satisfies
$$f'(z_0)=\sup_{g\in\mathcal{F}}|g'(z_0)|.$$
Try to explain the proof following this strategy as detailed as possible.

\bigskip

\noindent {\bf Problem 3.} Assume that u solves the nonlinear heat equation

$u_{t} = \frac{u_{xx}}{u_{x}^{2}} \mbox{ in } {\bf R} \times (0, \infty)$

with $u_{x} > 0.$ Find a transformation which changes the above equation into a linear PDE.

\bigskip

\noindent {\bf Problem 4.} Let $\phi\in C^\infty([0,T],\mathbb{R}^n)$ be a solution of linear wave equation
$$\sum_{\alpha,\beta=0}^ng^{\alpha\beta}\partial_\alpha\partial_\beta\phi=F,$$
where $\partial_\alpha:=\frac{\partial}{\partial x_\alpha}, \alpha=0,1,2,\cdots,n$, $t=x_0$,  $g^{\alpha\beta}, F$ are smooth functions in $[0,T]\times\mathbb{R}^{n}$, $g^{\alpha\beta}$ is symmetric and
$$\sup_{[0,T]\times\mathbb{R}^n}\sum_{\alpha,\beta}|g^{\alpha\beta}-\eta^{\alpha\beta}|<\frac{1}{2022},$$
where $\eta_{00}=-1, \eta_{ii}=1, i=1,2,\cdots,n$ and $\eta_{\alpha i}=0$ otherwise. Show that there is a constant $C$ depending only on $n$ such that for any $t\in[0,T]$,
$$\|\partial\phi(t,\cdot)\|_{L^2(\mathbb{R}^n)}\le C\left(\|\partial\phi(0,\cdot)\|_{L^2(\mathbb{R}^n)}+\int_0^t\|F(s,\cdot)\|_{L^2(\mathbb{R}^n)}\mathrm{d}s\right)$$
$$\times\exp\left(\int_0^t\sum_{\alpha,\beta}\|\partial g^{\alpha\beta}(s,\cdot)\|_{L^2(\mathbb{R}^n)}\mathrm{d}s\right),$$
here $|\partial f|^2:=\sum_{\alpha=0}^n|\partial_\alpha f|^2$.

\bigskip

\end{document}
