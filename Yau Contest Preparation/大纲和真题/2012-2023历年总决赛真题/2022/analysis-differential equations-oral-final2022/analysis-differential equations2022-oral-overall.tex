

\documentclass[12pt]{article}
\usepackage{amssymb}
\usepackage{amsmath}
\usepackage{graphics}
\usepackage{epsfig}
\usepackage{enumerate}
\usepackage{graphicx}
\usepackage{xcolor} % A package to add color.
%\usepackage{tensor}
\usepackage{geometry}
\usepackage{amsfonts,amssymb,amsmath}
\usepackage{slashed}
%\usepackage{tikz}
%\usetikzlibrary{arrows,calc}
%\usepackage{relsize}
%\usetikzlibrary{patterns}
%Andy's preamble

%Pictures
\usepackage{wrapfig}
\usepackage{tikz}
\usetikzlibrary{arrows,calc,decorations.pathreplacing}
%\definecolor{light-gray1}{gray}{0.90}
%\definecolor{light-gray2}{gray}{0.80}
\usepackage{cite}

\textwidth =15.6cm \textheight=22cm \hoffset0.5cm \voffset-1.3cm

\baselineskip=5mm
%\renewcommand{\baselinestretch}{1.3}
\renewcommand{\arraystretch}{1.5}
\oddsidemargin 0pt \evensidemargin 0pt
\parskip 0.1cm

\newtheorem{Assumption}{Assumption}[part]
\newtheorem{Corollary}{Corollary}[part]
\newtheorem{Definition}{Definition}[part]
\newtheorem{Example}{Example}[part]
\newtheorem{Lemma}{Lemma}[part]
\newtheorem{Proposition}{Proposition}[part]
\newtheorem{Remark}{Remark}[part]
\newtheorem{Theorem}{Theorem}[part]

\renewcommand{\theAssumption}{\thesection.\arabic{Assumption}}
\renewcommand{\theCorollary}{\thesection.\arabic{Corollary}}
\renewcommand{\theDefinition}{\thesection.\arabic{Definition}}
\renewcommand{\theequation}{\thesection.\arabic{equation}}
\renewcommand{\theExample}{\thesection.\arabic{Example}}
\renewcommand{\theLemma}{\thesection.\arabic{Lemma}}
\renewcommand{\theProposition}{\thesection.\arabic{Proposition}}
\renewcommand{\theRemark}{\thesection.\arabic{Remark}}
\renewcommand{\theTheorem}{\thesection.\arabic{Theorem}}

%\numberwithin{Assumption}{section} \numberwithin{Corollary}{section}
%\numberwithin{Definition}{section} \numberwithin{equation}{section}
%\numberwithin{Example}{section} \numberwithin{Lemma}{section}
%\numberwithin{Proposition}{section} \numberwithin{Remark}{section}
%\numberwithin{Theorem}{section}

\def\E{\mathbb E}


\pagestyle{empty}
\begin{document}

\begin{center}
S.-T. Yau College Student Mathematics Contests 2022\\

\vspace{0.1cm}

\Large {\bf Analysis and differential equations overall}

\vspace{0.1cm}

%\large {\bf Individual (5 problems)}

%\vspace{0.1cm}


\end{center}

\vskip 1cm


\noindent {\bf Problem 1.} Let $f: {\bf C} \rightarrow {\bf C}$ be a non-constant holomorphic function.

1) Prove that the image of $f$ is dense in ${\bf C}$.

2) Prove that the image of $f$ can miss only one point in ${\bf C}$.
   (Hint: The universal cover of ${\bf C} -\{0, 1\}$ is the unit disk.)

\bigskip

\noindent {\bf Problem 2.}  Let $K$ be a measurable function on $\mathbb{R}^n\times \mathbb{R}^n$. Define
$$Tf(x)=\int_{\mathbb{R}^n} K(x,y)f(y)\mathrm{d}y.$$
(1) Suppose that $K\in L^\infty_x L^1_y\cap L^\infty_yL^1_x$. Show that $T$ is a bounded operator on $L^2(\mathbb{R}^n)$.\\
Remark: $K\in L^\infty_x L^1_y$ means $\mathrm{ess\ sup}_{x\in\mathbb{R}^n}\int_{\mathbb{R}^n}|K(x,y)|\mathrm{d}y<+\infty$.\\
(2) Suppose that $K\in L^2(\mathbb{R}^n\times\mathbb{R}^n)$. Show that $T$ is a compact operator on $L^2(\mathbb{R}^n)$. \\
(3) Suppose that $K$ is compactly supported, and satisfies $|K(x,y)|\le A|x-y|^{-n+\alpha}$ for some $\alpha>0$, whenever $x,y\in\mathbb{R}^n$. Show that $K$ is not necessarily $\in L^2(\mathbb{R}^n\times\mathbb{R}^n)$, but  $T$ is still a compact operator on $L^2(\mathbb{R}^n)$.

\bigskip

\noindent {\bf Problem 3.} 
Consider the Cauchy problem for the linear homogeneous wave equation in ${\bf R}^{3}\times {\bf R}$:
\begin{align*}
	\Box\,\phi=0,\quad \phi(\bf x, 0)=\varphi(\bf x),\quad \partial_{t}\phi(\bf x,0)=\psi(\bf x).
\end{align*}
Suppose that the smooth functions $\varphi(\bf x), \psi(\bf x)$ have compact support and they only depend on the radial variable $r$, i.e. $\varphi(\bf x)=\varphi(r),\psi(\bf x)=\psi(r)$.
 \vspace{2mm}
 \begin{itemize}
 	\item The solution $\phi$ to the above Cauchy problem only depends on the radial variable $r$ and the time  variable $t$, i.e. $\phi(\bf x,t)=\phi(r,t)$.
 	\vspace{2mm}
 	
 	\item Prove that for sufficiently large $T_{0}>0$, we have
 	\begin{align*}
 		\partial_{r}\left(r\phi\right)(0,t)\equiv0,\quad \text{for all}\quad t\geq T_{0}.
 	\end{align*}
 \vspace{2mm}
 \item Let $u:=t-r,\, \bar{u}:=t+r$. Therefore $\phi$ can be viewed as a function of $(\bar{u},u)$. Prove that if there is a $r_{0}>0$ such that
 \begin{align*}
 	-\psi(r_{0})+\partial_{r}\varphi(r_{0})+\frac{1}{r_{0}}\varphi(r_{0})\slashed{=}0,
 \end{align*}
then there is a $u_{0}\in{\bf R}$ such that
\begin{align*}
	\lim_{\bar{u}\rightarrow\infty}\partial_{u}\left(r\phi\right)(\bar{u},u_{0})\slashed{=}0.
\end{align*}
 \end{itemize}
%\end{enumerate}


\end{document}
